\documentclass[letterpaper,11pt,leqno]{article}
\usepackage{microtype}
\usepackage[hyphens]{url}  % Load url package before biblatex to avoid option clash
\usepackage[style=mla,backend=biber]{biblatex}
\usepackage{paper,appendix}
\usepackage{graphicx}

% Configure biblatex formatting
\renewcommand*{\bibfont}{\small}
\setlength{\bibitemsep}{0pt}
\setlength{\bibhang}{\parindent}

% Enter paper title to populate PDF metadata:
\hypersetup{pdftitle={Minimalist LaTeX Template for Academic Papers}}

% Enter path to BibTeX file with references:

\addbibresource{bibliography.bib}

\begin{document}

% Enter title:
\title{PhilosophyHelperAI Methodology}

% Enter authors:
\author{Raghav Vikramprabhu, Amulya Jain
%
% Enter affiliations and acknowledgements:
\thanks{Raghav Vikramprabhu: Georgia Institute of Technology. Amulya Jain: Georgia Institute of Technology. We thank Google for releasing their paper on how to properly prompt an AI.}}

% Enter date:
\date{June 2025}   

% Enter permanent URL (can be commented out):
% \available{https://github.com/pmichaillat/latex-paper}

\begin{titlepage}
\maketitle

% Enter abstract:
This is the abstract. 

\end{titlepage}


% Enter main text:
\section{Introduction}\label{s:introduction}
 
During our final project brainstorming session for Environmental Ethics, we wanted a way to incorporate Generative AI in our final project because were were interested in it. The way we settled on was PhilosophyHelperAI, a handy tool that helps you think (not think for you). \cite{GoogleAIDev}


\section{Methodology}\label{s:section}

The core Methodology of our project was human autonomy and dignity.
This is the reason why we choose to have the agent only respond in questions rather than definite answers, to the users' dismay:

\begin{figure}[h!]
  \centering
  \includegraphics[width=\textwidth]{docs/images/userresponse1.png}
  \caption{This is an example of the agent only responding in questions in order to help force the user to ponder.}
  \label{fig:sample}
\end{figure}

\cite{PromptEngineering}

\section{Challenges}\label{s:section}

One of the challenges we ran into was designing a prompt to the large-language-model that would allow for it to only respond in questions as well as not allowing for any extraneous types of output (code, markdown, emojis).

Our final prompt can be seen here:

\begin{figure}[h!]
  \centering
  \includegraphics[width=\textwidth]{docs/images/prompt.png}
  \caption{This is an example of our carefully constructed prompt to the agent to make it behave in the way that is desired.}
  \label{fig:sample}
\end{figure}

\subsection{Generic subsection}

Lorem ipsum dolor sit amet, consectetur adipiscing elit. Nulla facilisi. Nullam molestie, libero sit amet luctus vehicula, eros purus ultrices libero, eget fermentum leo sapien a metus. Duis porta massa vel justo posuere, nec placerat neque dictum. Morbi nec velit in turpis fermentum cursus nec ut leo. Integer vitae eros vehicula, fermentum turpis sed, fermentum elit. Suspendisse ac mauris at nisl ultricies commodo id nec justo. 

\pagebreak

\printbibliography

\end{document}